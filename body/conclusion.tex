
%=========================================================================================
\BiChapter{结论与展望}{}

\BiSection{结论}{Subequations}


目前针对对话场景的ECPE研究较少,现有方法性能难以令人满意。在先前的ECPE研究中,研究者设置了上界实验,以探索子任务间共享预测效果。然而,这些方法易受错误传递问题影响,从而影响最终性能。
为了解决这个问题并提升提取情感原因对效果,我们提出了一种名为Dag-Rank的一步方法来进行情感-原因对提取。该方法包括以下步骤:学习子句向量表示;通过有向无环图(DAG)模拟子句间关系,获得更好的子句表示;将相对位置增强的子句对排序整合到一个统一的神经网络中,并以端到端的方式对子句对候选项进行排名以提取情感原因对。与先前的工作相比,我们的方法具有两个主要优势:彻底避免了错误传递问题,使得模型更为合理和现实;充分考虑了对话场景下的先验知识,为每个话语寻找了更贴切的情感提取,而不仅仅是将每个话语与一定数量的周围话语连接起来,因此能更好地提取信息。

 本文在RECCON-DD数据集上进行了大量的实验。实验结果表明,Dag-Rank方法在多个任务上均表现出色,且在提取情感原因对、情感子句和原因子句方面优于其他基线系统。与其他现有方法相比,我们的方法在提取正确的情感原因对方面具有更高的召回率和相近的精确度,表明我们的一步方法在情感原因提取任务上具有很大的潜力。
此外,我们还对模型的各个部分进行了详细的消融实验、子任务监督信号比较、DAG结构影响分析以及排序方法比较。这些实验进一步验证了我们提出的方法在处理错误传递问题和利用有向无环图捕捉子句间关系方面的独特优势。
总之,我们提出的Dag-Rank方法在抽取情感原因对方面取得了显著的性能改进,其方法在处理错误传递问题和利用有向无环图捕捉子句间关系方面具有独特的优势,为情感原因对提取任务提供了一个有效的解决方案。

\BiSection{展望}{Subequations}

目前关于情感原因分析的研究主要集中在较为粗粒度的子句级提取上,有必要进一步设计细粒度的方法,可以提取跨度级别或短语级别的情感表达和原因。其次,设计有效的方法将适当的语言知识注入到神经模型中,对情感分析任务具有价值。此外,探索情感的语义角色将是一个有趣且具有挑战性的方向。这需要考虑情感表达的完整结构,包括主观体验者、情感对象和情感原因等,从而更深入地理解情感语境。我们还可以视图将可以将本方法应用于其他领域,如事件抽取、知识蒸馏和多任务学习,以充分发挥其在自然语言处理任务中的潜力。
通过对上述方向的探索和研究,我们期望在情感原因对提取任务以及其他相关自然语言处理任务中取得更好的性能和应用效果。

