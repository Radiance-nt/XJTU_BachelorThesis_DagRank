% !Mode:: "TeX:UTF-8" 

\BiChapter{格式要求}{Formats}

\BiSection{封面}{Cover}

采用西安交通大学毕业设计(论文)统一封面(模板见毕业设计管理系统——资料下载), 用 A3 纸(封面封底连在一起,从左侧包住论文)。封面上所填内容居中排列三字号加粗,论 文题目不能超过 35 个汉字。封面可去教材科购买。


\BiSection{任务书、考核评议书}{}

双面打印。《考核评议书》、《评审意见书》和《答辩结果》必须分别由 指导教师、评阅人和答辩组据实填写。

\BiSection{中、英文摘要页}{Abstracts}

摘要页由摘要正文、关键词等组成。

摘要/ABSTRACT 按一级标题编排中文“摘要”二字,二字间距为两个字符。英文为“ABSTRACT”。摘要正文,中文每段开头左起空两字符起排,段与段之间不空行;英文每段开头左对齐顶格编
排,段与段之间空一行。小四号字。 

关键词/KEY WORDS” 正文内容下,空一行,左对齐顶格编排“关键词/KEY WORDS”(小四号,加粗),后接冒号, 其后为具体关键词(小四号),每一关键词之间用分号分开,最后一个关键词后无标点符号。英文 每组KEY WORDS 的第一个字母为大写,其余为小写。

资助申明:资助申明编排在摘要第一页的页脚处。

\BiSection{中、英文目录页}{Contents}
中文目录页应放在奇数页上起排。 “目录”二字按一级标题编排,两字间距两个字符。 目录正文,包括编号、标题及其开始页码。一般只列到三级标题。目录中标题的编号应与正文
中标题的编号一致; 第一级标题左对齐顶格编排;与上一级标题相比,下一级标题左缩进一个字符起排; 标题与页码之间用“……”连接。页码右对齐顶格编排; 建议采用文本编辑软件的“目录自动生成功能”生成目录。 如果有英文目录,英文目录的内容、格式均须与中文目录一致。

\BiSection{表格}{Tables}
图、表、公式等的序号用阿拉伯数字分章连续编号,如图 \ref{fig_ch2}、表 \ref{tab_ch2}、表 \ref{tab_check} 等,但
不出现“公式"两字,将编号置入小括号中,如(\ref{equ_ch2_fourier})等。图、表和公式等与正文之间间隔 0.5 行。 图应有图题,表应有表题,并分别置于图号和表号之后。图号和图题置于图下方的居中位置,
表号和表题应置于表上方的居中位置。引用图或表应在图题或表题右上角标出文献来源。 若图或表中有附注,采用英文小写字母顺序编号,附注写在图或表的下方。 物理量及量纲均按国际标准(SI)及国家规定的法定符号和法定计量单位标注,禁止使用已废
弃的符号和计量单位。物理量的符号由斜体字母标注,单位的符号使用正体字母标注,量与单位间 用斜线隔开。例如: $I/A$,$\rho/kg \cdot m^{-3}$,$F/N$,$v/m \cdot s^{-1}$ 等等。

表格应紧跟文述编排。表格中一般是内容和测试项目由左至右横读,数据依序竖读,应有自明 性。表的各栏均应标明“量或测试项目、符号、单位”。只有在无必要标注的情况下方可省略。表内 同一栏的数字必须上下对齐。表内不宜用“同上”、“同左”和类似词,一律填入具体数字或文字。表内“空 白”代表未测或无此项,“…”代表未发现,“0”代表实测结果确为零。如数据已绘成曲线图,可不再列 表。

\begin{enumerate}[wide, label=\arabic*),  labelindent=21pt]
    \item 表格转页接排时,在随后的各页上应重复表的编号。编号后跟(续),如表 l(续),续表
均应重复表头和关于单位的陈述。
    \item 一律使用三线表,与文字齐宽,上下边线,线粗 1.5 磅,表内线,线粗 1 磅。在三线表中可以加辅助线,以适应较复杂表格的需要。
    \item 使用他人表格须注明出处。
    \item 表中用字一般为五号字。如排列过密,用五号字有困难时,可小于五号字,但不小于七号
\end{enumerate}


\BiSection{图片}{Figures}

\begin{enumerate}[wide, label=\arabic*),  labelindent=21pt]
    \item 一幅图如有若干幅分图,均应编分图号,用(a),(b),(c),…按顺序编排;
    \item 插图须紧跟文述。在正文中,一般应先见图号及图的内容后见图,特殊情况须延后的插图
不应跨节; 
    \item 提供照片应大小适宜,主题明确,层次清楚,利于复制,金相照片一定要有比例尺;
    \item 图应具有"自明性”,即只看图、图题和图例,不阅读正文,就可理解图意。
    \item 图中的标目是说明坐标轴物理意义的项目,由物理量的符号或名称和相应的单位组成。
    \item 图中用字一般为五号字,如排列过密,用五号字有困难时,可小于五号字,但不得小于七
号字。 
    \item 图的大小一般为宽 6.67cm $\times$ 高 5.00cm。特殊情况下,也可宽 9.00cm $\times$ 高 6.75cm,或宽 13.5cm
$\times$ 高 9.00cm。同类图片的大小应一致。图片的编排应美观、整齐。
\end{enumerate}


\BiSection{其他}{Others}

\BiSubsection{页码}{}

论文页码的第一页\textbf{从正文开始}用\textbf{阿拉伯数字}标注,直至全文结束。
\textbf{正文前的内容}(除 封面)用\textbf{罗马数字}单独标注页码。
页码位于页面底端,对齐方式为 “外侧”,页码格式为最简单的数字,不带任何其它的符号或信息。
页码不能出现缺页和重复页。附录(含外文原文及其译文、有关图纸、计算机源程序等)必须与论文装订在一起,附录的页码必须接着参考文献的页码连续编写。